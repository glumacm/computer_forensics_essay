\documentclass[12pt,a4paper,openany]{article}

%Uporabljeni paketi
\usepackage[utf8]{inputenc}
\usepackage{cmap}
\usepackage{type1ec}
\usepackage[T1]{fontenc}
\usepackage{fancyhdr}
\usepackage{graphicx,epsfig}
\usepackage[slovene]{babel}
\usepackage{cite}

\usepackage[pdftex,colorlinks,citecolor=black,filecolor=black,linkcolor=black,urlcolor=black,pagebackref]{hyperref}
\usepackage{tikz}

\title{Bitcoin Block Withholding Attack : Analiza in ublažitev napada}
\author{ Anej Budihna,Luka Golinar, Matjaz Glumac  }

\begin{document}
\maketitle
\section{Uvod}
\paragraph{}
Avtorji se posvetijo dvema problemoma: prvi je študija različice napada imenovanega "bločno prikrivanje" (angl. block withholding attack - BWA) v Bitcoinih in drugi je priporočilo rešitev za preprečitev vseh obstoječih vrst  BWA napadov. Predstavijo analize strategij sebičnega Bitcoin rudarja, ki v potuhi z neko skupino rudarjev poskuša napadati neko drugo skupino in pri tem dobi določeno nagrado, ker je prisostvoval pri napadu na drugo skupino. Tak napad so avtorji poimenovali "sponzorirani napad z bločnim prikrivanjem". Poleg tega predstavijo podrobno kvantitivno analizo dobičkonosnosti, ki jo lahko sebični rudar pridobi s tem, ko rudar uporablja omenjen napad v različnih primerih. V članku avtorji dokažejo, da ob določenih pogojih lahko napadalec optimalno poveča svoj prihodek z uporabo nekaterih strategij in s pametnim izkoriščanjem svojih računalniških virov. Pokažejo tudi, da lahko napadalec uporabi to strategijo za napad na obe skupini, da bi pri tem lahko dosegel višjo dobičkonosnost.
\newline
\indent Najpomembneje, predstavijo strategijo, ki se lahko učinkotivto zoperstavi napadu bločnega prikrivanja v katerikoli rudarski skupini. Prvo priporočajo generično shemo, ki uporablja kriptografsko zavezujoče sheme za zoperstavitvi takemu napadu. Nato priporočajo alternativno implementacijo enake sheme z uporabo razpršilnih (angl. hash) funkcij. Taka shema ščiti skupino rudarjev pred zlonamernimi rudarji, tako navadnimi kot tudi administratorskimi redarji. Tako ta shema kot tudi druge njene različice ponujajo obrambo pred BWH napadi s tem, da onemogočijo možnost, da rudarji razlikujejo med celovitimi in delnimi dokazi dela. Prav tako pa te sheme omogočajo zaščito, da administratorji ni omogočeno goljufanje znotraj skupine katero nadzorujejo. Shemo se lahko implementira tako, da se napravi korenito spremembo na obstoječem Bitcoin protokolu. Na koncu se tudi posvetijo analizi varnosti opisane sheme.



\end{document}
\begin{document}

\end{document}